\documentclass[12pt]{scrartcl}

% LaTeX Template für Abgaben an der Universität Stuttgart
% Autor: Sandro Speth
% Bei Fragen: Sandro.Speth@iste.uni-stuttgart.de
%-----------------------------------------------------------
% Modul fuer verwendete Pakete.
% Neue Pakete einfach einfuegen mit dem \usepackage Befehl:
% \usepackage[options]{packagename}
\usepackage[utf8]{inputenc}
\usepackage[T1]{fontenc}
\usepackage[ngerman]{babel}
\usepackage{lmodern}
\usepackage{graphicx}
\usepackage[pdftex,hyperref,dvipsnames]{xcolor}
\usepackage{listings}
\usepackage[a4paper,lmargin={2cm},rmargin={2cm},tmargin={3.5cm},bmargin = {2.5cm},headheight = {4cm}]{geometry}
\usepackage{amsmath,amssymb,amstext,amsthm}
%\usepackage[lined,algonl,boxed]{algorithm2e}
% alternative zu algorithm2e:
%\usepackage[]{algorithm} %counter mit chapter
\usepackage{algpseudocode}
\usepackage{tikz}
\usepackage{hyperref}
\usepackage{url}
\usepackage[inline]{enumitem} % Ermöglicht ändern der enum Item Zahlen
\usepackage[headsepline]{scrlayer-scrpage} 
\pagestyle{scrheadings} 
\usetikzlibrary{automata,positioning}
\usepackage{bookmark}
\usepackage{multicol}
\usepackage{pdfpages}
\usepackage[T1]{fontenc}
\usepackage{mnsymbol}
% LaTeX Template für Abgaben an der Universität Stuttgart
% Autor: Sandro Speth
% Bei Fragen: Sandro.Speth@iste.uni-stuttgart.de
%-----------------------------------------------------------
% Modul beinhaltet Befehl fuer Aufgabennummerierung,
% sowie die Header Informationen.

% Überschreibt enumerate Befehl, sodass 1. Ebene Items mit
\renewcommand{\theenumi}{(\alph{enumi})}
% (a), (b), etc. nummeriert werden.
\renewcommand{\labelenumi}{\text{\theenumi}}

\renewcommand{\theenumii}{\roman{enumii}}
% (a), (b), etc. nummeriert werden.
\renewcommand{\labelenumii}{\text{\theenumii}.}

%Algorithmus umstylen
\renewcommand{\algorithmicfor}{LOOP}
\renewcommand{\algorithmicwhile}{WHILE}
\renewcommand{\algorithmicdo}{DO}
\renewcommand{\algorithmicend}{END}
\renewcommand{\algorithmicif}{IF}
\renewcommand{\algorithmicthen}{THEN}


% Counter für das Blatt und die Aufgabennummer.
% Ersetze die Nummer des Übungsblattes und die Nummer der Aufgabe
% den Anforderungen entsprechend.
% Gesetz werden die counter in der hauptdatei, damit siese hier nicht jedes mal verändert werden muss
% Beachte:
% \setcounter{countername}{number}: Legt den Wert des Counters fest
% \stepcounter{countername}: Erhöht den Wert des Counters um 1.
\newcounter{sheetnr}
\newcounter{exnum}

% Befehl für die Aufgabentitel
\newcommand{\exercise}[1]{\section*{Aufgabe \theexnum\stepcounter{exnum}: #1}} % Befehl für Aufgabentitel

% Formatierung der Kopfzeile
% \ohead: Setzt rechten Teil der Kopfzeile mit
% Namen und Matrikelnummern aller Bearbeiter
\ohead{Johannes Heugel (3660353) \\
Julia Waedt (3654521)}
\chead{Abgabe}
% \ihead: Setzt linken Teil der Kopfzeile mit
% Modulnamen, Semester und Übungsblattnummer
\ihead{Theoretische Informatik II\\
Sommersemester 2024\\
Übungsblatt \thesheetnr}

\setcounter{sheetnr}{3}
\setcounter{exnum}{2}

\begin{document}

\exercise{}
\begin{enumerate}
    \item \begin{enumerate}
        \item $E \leq Q$\\
              Seien v, u Kodierungen für zwei Turingmaschinen.\\\\
              \begin{tabular}[h!]{l}
                \hline
                Definition von $M_{u, v}$\\\hline
                1: \textbf{input $x \in \{0,1\}^*$}\\
                2: Simuliere $M_u$ auf Eingabe x\\
                3: Simuliere $M_v$ auf Eingabe x\\
                4: Falls beide akzeptieren: \textbf{reject}\\
                5: Falls beide nicht akzeptieren: \textbf{reject}\\
                6: Sonst: \textbf{accept}\\\hline
              \end{tabular}\\\\
              Definiere $f: \, \{0,1\}^* \rightarrow \{0,1\}^* \# \{0,1\}^*$ mittels $f(w) := v\#w$\\\\
              Zu zeigen: $w \in E \Longleftrightarrow f(w) \in Q $ für $w \in \{0,1\}^*$\\\\
              Zu $\Rightarrow$: Sei $w \in E$. Dann hält $M_w$ auf keiner Eingabe. Also hält 
              $M_{u,v}$ auf keinem $x \in \{0,1\}^*$. Dann halten $M_u$ und $M_v$ entweder beide auf jedem x oder beide halten auf keinem.
               Also gilt $x \in T(M_u) \cap T(M_v)$ oder $x \notin T(M_u) \cap T(M_v)$ für jedes $x \in \{0,1\}^*$. Es ist somit
               $ T(M_u) = T(M_v)$, also $f(w) \in Q$ für jedes $x \in \{0,1\}^*$.\\
 
               Zu $\Leftarrow$: Sei $w \notin E$. Dann hält $M_w$ auf mindestens einer Eingabe. Also
               $M_{u,w}$ für alle Eingaben. Dann akzeptiert jeweils nur eine der Turingmaschinen $M_u$ und 
               $M_v$ für die Eingabe $x$. Es gilt also entweder $x \in T(M_u)$ und $x \notin T(M_v)$ oder anders herum.
               Also $T(M_u) \neq T(M_v)$ und somit $f(w) \notin Q$.
        \item $H_0 \leq U$
              Sei $f(w)$ die Kodierung einer Turingmaschine $M_{f(w)}$.\\\\
              \begin{tabular}[h!]{l}
                \hline
                Definition von $M_{f(w)}$\\\hline
                1: \textbf{input $x \in \{0,1\}^*$}\\
                2: Simuliere $M_{w}$ auf leerer Eingabe.\\
                3: \textbf{accept}\\\hline
              \end{tabular}\\\\
              Definiere $f: \, \{0,1\}^* \rightarrow \{0,1\}^*$ mittels $f(w) := w^{'}$ ?\\\\
              Zu zeigen: $w \in H_0 \Longleftrightarrow f(w) \in U $ für $w \in \{0,1\}^*$\\
              Zu $\Rightarrow$: Sei $w \in H_0$. Dann hält $M_{w}$ auf leerer Eingabe. Also hält $M_{f(w)}$ auf jedem $x \in \{0,1\}^*$.
              Somit ist $T(M_{f(w)}) = \Sigma^*$, also $f(w) \in U$\\

              Zu $\Leftarrow$: Sei $w \notin H_0$. Dann hält $M_{w}$ nicht auf leerer Eingabe. Also hält $M_{f(w)}$ auf keinem $x \in \{0,1\}^*$.
              Somit ist $T(M_{f(w)}) = \emptyset$, also $f(w) \notin U$\\
\newpage
        \item $U \leq Q$\\
              Seien v, u Kodierungen für zwei Turingmaschinen.\\\\
              \begin{tabular}[h!]{l}
                \hline
                Definition von $M_{u, v}$\\\hline
                1: \textbf{input $x \in \{0,1\}^*$}\\
                2: Simuliere $M_u$ auf Eingabe x\\
                3: Simuliere $M_v$ auf Eingabe x\\
                4: Falls beide akzeptieren: \textbf{accept}\\
                5: Falls beide nicht akzeptieren: \textbf{accept}\\
                6: Sonst: \textbf{reject}\\\hline
              \end{tabular}\\\\
              Definiere $f: \, \{0,1\}^* \rightarrow \{0,1\}^* \# \{0,1\}^*$ mittels $f(w) := v\#w$\\\\
              Zu zeigen: $w \in U \Longleftrightarrow f(w) \in Q $ für $w \in \{0,1\}^*$\\

              Zu $\Rightarrow$: Sei $w \in E$. Dann hält $M_w$ auf jeder Eingabe. Also hält 
              $M_{u,v}$ auf jedem $x \in \{0,1\}^*$. Dann halten $M_u$ und $M_v$ entweder beide auf jedem x oder beide halten auf keinem.
               Also gilt $x \in T(M_u) \cap T(M_v)$ oder $x \notin T(M_u) \cap T(M_v)$ für jedes $x \in \{0,1\}^*$. Es ist somit
               $ T(M_u) = T(M_v)$, also $f(w) \in Q$ für jedes $x \in \{0,1\}^*$.\\
 
               Zu $\Leftarrow$: Sei $w \notin E$. Dann hält $M_w$ auf mindestens einer Eingabe nicht. Also
               $M_{u,w}$ für mindestens eine Eingabe nicht (?). Dann akzeptiert jeweils nur eine der Turingmaschinen $M_u$ und 
               $M_v$ für die Eingabe $x$. Es gilt also entweder $x \in T(M_u)$ und $x \notin T(M_v)$ oder anders herum.
               Also $T(M_u) \neq T(M_v)$ und somit $f(w) \notin Q$.
        \item $I \leq U$\\
        Sei $f(w)$ die Kodierung einer Turingmaschine $M_{f(w)}$.\\\\
        \begin{tabular}[h!]{l}
          \hline
          Definition von $M_{f(w)}$\\\hline
          1: \textbf{input $x \in \{0,1\}^*$}\\
          2: Simuliere $M_{w}$ auf Eingabe x.\\
          3: \textbf{accept}\\\hline
        \end{tabular}\\\\
        Definiere $f: \, \{0,1\}^* \rightarrow \{0,1\}^*$ mittels $f(w) := w^{'}$ ?\\\\
        Zu zeigen: $w \in H_0 \Longleftrightarrow f(w) \in U $ für $w \in \{0,1\}^*$\\
        Zu $\Rightarrow$: Sei $w \in I$. Dann hält $M_{w}$ auf unendlich vielen Eingabe. Also hält $M_{f(w)}$ auf jedem $x \in \{0,1\}^*$.
        Somit ist $T(M_{f(w)}) = \Sigma^*$, also $f(w) \in U$\\

        Zu $\Leftarrow$: Sei $w \notin I$. Dann hält $M_{w}$ nicht auf leerer Eingabe. Also hält $M_{f(w)}$ auf keinem $x \in \{0,1\}^*$.
        Somit ist $T(M_{f(w)}) = \emptyset$, also $f(w) \notin U$\\
    \end{enumerate}
    \item $H_0$ ist semi-entscheidbar, aber nicht co-semi-entscheidbar. $E$ ist unentscheidbar
\end{enumerate}

\newpage

\setcounter{exnum}{4}

\exercise{}
\begin{enumerate}
  \item Sei \small{$F:= (\exists w(1+w=m)) \wedge (\exists w^{'}(m+w^{'}+1=n)) \wedge (\forall v(\exists k(v*k=m) \wedge \exists k(v*k=n)) \rightarrow v \leq 1)$}\\
  
        Für $m, n \in \mathbb{N}$ wird 
        \begin{align*}
          F(m, n) &\Longleftrightarrow m \, \mathrm{liegt \, zwischen \, 1 \, und} \,n-1 \mathrm{und \, jede \, Zahl, \,die} \, m \, \mathrm{und}\, n\, \mathrm{teilt, \,ist\, maximal} \, 1\\
                  &\Longleftrightarrow 1 \leq m < n \wedge ggT(m,n) = 1\\
                  &\Longleftrightarrow f(n)=m
        \end{align*}
  \item 
\end{enumerate}

\end{document}